% Hier wird die Dokumentenklasse aus dem übergeordneten Verzeichnis geladen
% !!!!!!!! Diese Datei sollte möglichst so bestehen bleiben. Wird für ein
% einzelnes Protokoll ein extra Paket benötigt, sollte dies in der dazugehörigen
% protokoll.text Datei hinzugefügt werden.
\documentclass[ngerman,leqno,x11names,fleqn]{uni-abgabe}

% Hier werden die Namen der Teammitglieder eingestellt
\partner{Jan Eike}{Suchard}{jan.eike.suchard@uni-oldenburg.de}{5945617}
\partner{Michaela}{Schammler}{michaela.schammler@uni-oldenburg.de}{XXXXXXX} %TODO: Matrikelnummer anpassen

% Hier werden die Daten über das Modul eingestellt
\module{phy214}{Experimentalpraktikum Physik}
\semester{Wintersemester 2023/2024}

% Hier wird die Nutzung von BibLaTeX als Quellenverwaltung eingerichtet
\usepackage{csquotes}
\usepackage[backend=biber,natbib=true,style=numeric,sorting=none,backref=false,alldates=iso,seconds=true,block=nbpar]{biblatex}
\addbibresource{../global-sources.bib}
\setlength{\bibitemsep}{1.5em}


% Hier wird eingestellt, dass Bilder entweder aus dem images der auf der 
% obersten Verzeichnisebene liegt geladen werden können, oder aus dem images
% Verzeichnis des Protokolls
\graphicspath{{./images/},{../images/}}

% Mit diesem Package ist es möglich das Protokoll auf mehrere einzelstehende
% LaTeX Dateien aufzuteilen, die eigentständig kompilierbar sind
\usepackage{subfiles}