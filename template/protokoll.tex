% !TeX root = ./protokoll.tex
% Hier wird die Dokumentenklasse aus dem übergeordneten Verzeichnis geladen
% !!!!!!!! Diese Datei sollte möglichst so bestehen bleiben. Wird für ein
% einzelnes Protokoll ein extra Paket benötigt, sollte dies in der dazugehörigen
% protokoll.text Datei hinzugefügt werden.
\documentclass[ngerman,leqno,x11names,fleqn]{uni-abgabe}

% Hier werden die Namen der Teammitglieder eingestellt
\partner{Jan Eike}{Suchard}{jan.eike.suchard@uni-oldenburg.de}{5945617}
\partner{Michaela}{Schammler}{michaela.schammler@uni-oldenburg.de}{XXXXXXX} %TODO: Matrikelnummer anpassen

% Hier werden die Daten über das Modul eingestellt
\module{phy214}{Experimentalpraktikum Physik}
\semester{Wintersemester 2023/2024}
\group{Nachmittags (13-17 Uhr)}
\team{XX} %TODO: Anpassen

% Hier wird die Nutzung von BibLaTeX als Quellenverwaltung eingerichtet
\usepackage{csquotes}
\usepackage[backend=biber,natbib=true,style=numeric,sorting=none,backref=false,alldates=iso,seconds=true,block=nbpar]{biblatex}
\addbibresource{../global-sources.bib}
\setlength{\bibitemsep}{1.5em}


% Hier wird eingestellt, dass Bilder entweder aus dem images der auf der 
% obersten Verzeichnisebene liegt geladen werden können, oder aus dem images
% Verzeichnis des Protokolls
\graphicspath{{./images/},{../images/}}

% Mit diesem Package ist es möglich das Protokoll auf mehrere einzelstehende
% LaTeX Dateien aufzuteilen, die eigentständig kompilierbar sind
\usepackage{subfiles}

% TODO: Titel anpassen an den Titel des aktuellen Versuches
\title{Vorlagenprotokoll \texorpdfstring{{\color{red} änder mich!}}{Änder mich}}

% TODO: Dieses Datum auf das Versuchsdatum einstellen
\date{\today}

% TODO: Dieses Datum auf das Datum einstellen, an dem das Protokoll abgegeben 
% werden soll.
\submissiondate{\today}

% TODO: Dieses Paket stellt einen Demotext bereit, es sollte entfernt werden und
% die Inhalte in das Protokoll geschrieben werden
\usepackage[math]{blindtext}

\begin{document}
\maketitle

% Hier wird der Zeilenabstand auf das Äquivalent von Words 1,5-Zeilig gesetzt
\doublespacing

% Hier wird die Unterdatei der Einleitung in das Protokoll geladen und
% anschließend wird die Seite umgebrochen
\subfile{parts/introduction.tex}
\newpage
% Hier wird die Unterdatei des ersten Versuchs in das Protokoll geladen und
% anschließend wird die Seite umgebrochen
\subfile{parts/experiment1.tex}
\newpage

% Hier wird nun der Zeilenabstand auf einzeilig zurückgesetzt, da nun der Anhang
% folgt und dieser kompakt sein sollte.
\singlespacing
\part{Anhang}
\blindtext
\printbibliography[heading=bibnumbered,title=Referenzen und Literatur]
\newpage

\section{Messwerttabelle}
Die Messwerttabelle befindet sich auf den folgenden Seiten.
Auf die Messwerttabelle folgen keine weiteren Anhänge.

% Hier werden alle Seiten aus der Datei "messwerttabelle.pdf" angehängt.
% Da das Bauen des Dokumentes fehlschlägt, wenn die Datei nicht vorhanden ist,
% ist die folgende Zeile auskommentiert und muss wieder einkommentiert werden
% damit die Messwerttabelle angehängt wird.
% \includepdf[pages=-]{messwerttabelle.pdf}


\end{document}